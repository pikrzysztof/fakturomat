\documentclass[12pt]{article}
\usepackage[MeX]{polski}
\usepackage[latin2]{inputenc}

\setlength{\marginparwidth}{0pt}
\setlength{\parindent}{0pt}
\addtolength{\hoffset}{-50pt}
\pagestyle{empty}

% Tu wpisa� dane do faktury!
\newcommand{\datasprzedazy}{3.03.2006}
\newcommand{\terminplatnosci}{17.03.2006}
\newcommand{\nrfaktury}{3/2006} 
\newcommand{\netto}{1000.00} 
\newcommand{\vat}{220.00}
\newcommand{\brutto}{1220.00}
\newcommand{\slownie}{jeden tysi�c dwie�cie dwadzie�cia z�otych}
% Koniec danych do faktury

\begin{document}

 \begin{tabular}{p{0.8\textwidth} l l}
  \textit{Sprzedawca:} & Data sprzeda�y: & \datasprzedazy \\
   & Data wystawienia: & \datasprzedazy \\
  Jan Kowalski & & \\
  ul. U�ytkownik�w Windy 283/9999999 & & \\
  50-000 Wroc�aw & & \\
  NIP 123-456-78-90 & &
 \end{tabular}
 \ \\ \ \\
 \centerline{\hspace{50pt}\LARGE{Faktura VAT nr \nrfaktury}}\\
 \centerline{\hspace{50pt}(orygina� / kopia)}\\
 \ \\
 \textit{Nabywca:} \\
 \ \\
 Micro\$oft Corp. \\
 ul. Gdzie� w Redmond \\
 Gdzie� w USA \\

 \begin{tabular}{r l l r p{2cm} p{2cm} l r p{2cm}}
  \hline 
  Lp & Nazwa us�ugi & j.m. & Ilo�� & Cena jedn.\newline netto & Warto�� netto & VAT & Podatek & Warto�� z podatkiem \\ 
  \hline
  1 & Straty moralne & szt & 1 & \netto & \netto & 22 \% & \vat & \brutto \\
  \hline
  & & & & Razem: & \netto & & \vat & \brutto \\
  & & & & W tym: & \netto & 22 \% & \vat & \brutto \\
  \\
  
 \end{tabular}

 \ \\
 S�ownie: \slownie\\
 \ \\
 Forma p�atno�ci: PRZELEW \\
 Termin p�atno�ci: \terminplatnosci \\


 \ \\ \ \\ \ \\
 ..............................\hspace{220pt}..............................\\
 \ \ \ Osoba odbieraj�ca\hspace{225pt}\mbox{Osoba wystawiaj�ca}

\end{document}
